\documentclass[12pt]{homework}
\newcommand{\transprob}{\P{s_i}{s_{i-1}}}
\newcommand{\obsprob}{\P{w}{s}}
\title{}
\author{}
\date{}
\begin{document}
\maketitle
\header{CS4248}{Assignment 2}{U096883L Shawn Tan}
\section{Introduction}

This assignment requires us to build a Part-of-Speech (POS) tagger, using 
training data from part of the Penn Treebank. The method used in our approach 
has to employ a Hidden Markov Model (HMM). This entails learning from the 
training data a set of parameters required for the HMM. Table \ref{parameters} 
shows the various sets of data we have to collect for this particular 
assignment.

In the following sections, we will explain in detail how individual aspects of 
the HMM was created, and experiment with two simple smoothing techniques, 
Laplace and Witten-Bell smoothing. The two techinques will be evaluated 
according to their precision, recall and F1 measures.

\begin{table}
	\begin{center}
	\begin{tabular}{l l}
		\hline
		Name			&	Description\\
		\hline
		$V$				&	all unique words\\
		$S$				&	all unique POS tags\\
		$\transprob$	&	transition probability from one POS tag to another\\
		$\obsprob$		&	probability of seeing a word given a POS tag\\
		\hline
	\end{tabular}
	\end{center}
	\label{parameters}\caption{Parameters for a POS tagger HMM}
\end{table}
\end{document}
