\documentclass[12pt]{homework}
\newcommand{\transprob}{\P{s_i}{s_{i-1}}}
\newcommand{\obsprob}{\P{w}{s}}
\title{}
\author{}
\date{}
\begin{document}
\maketitle
\header{CS4248}{Assignment 2}{U096883L Shawn Tan}
\section{Introduction}

This assignment requires us to build a Part-of-Speech (POS) tagger, using 
training data from part of the Penn Treebank. The method used in our approach 
has to employ a Hidden Markov Model (HMM). This entails learning from the 
training data a set of parameters required for the HMM. Table \ref{parameters} 
shows the various sets of data we have to collect for this particular 
assignment.

In the following sections, we will explain in detail how individual aspects of 
the HMM was created, outlining some of the technical difficulties faced. We also  
experiment with two simple smoothing techniques, Laplace (add-one) and 
Witten-Bell smoothing. The two techniques will be evaluated according to their 
precision, recall and F1 measures.

\begin{table}
	\begin{center}
	\begin{tabular}{l l}
		\hline
		Name			&	Description\\
		\hline
		$V$				&	all unique words\\
		$S$				&	all unique POS tags\\
		$\transprob$	&	transition probability from one POS tag to another\\
		$\obsprob$		&	probability of seeing a word given a POS tag\\
		\hline
	\end{tabular}
	\end{center}
	\label{parameters}\caption{Parameters for a POS tagger HMM}
\end{table}
\section{Learning from \texttt{sents.train}}
The \texttt{sents.train} dataset contains 39,832 lines, each word annotated with 
POS tags. In order to extract the relevant information we count the transitions.

In order to obtain $\transprob$, we need to count the different number of times 
one tag is followed by another. For each line, we extract the POS for each 
token, leaving a list of POS tags. We then prepend a `{\char`\^}' at the 
beginning, and a `\$' character at the end. This ensures that the probabilities 
for a given POS starting a sentence are also taken into account.

Another preprocessing step performed was to change all instances of numeric 
tokens, and replacing all digits with \textbackslash\#. For example, `\$25.50' 
will be converted to 
`\$\textbackslash\#\textbackslash\#.\textbackslash\#\textbackslash\# '. This 
reduces the many different combinations of numbers down to a common token that 
would have a higher probability given the tag CD.

Going through the file, we maintain a dictionary of the following items:
\begin{center}
	\begin{tabular}{l p{10cm}}
	\hline
	Name	&	Description\\
	\hline
	$C(q_{t-1},q_t)$	& The number of times a POS tag occurs at time $t$ 
	after a POS tag occurs at time $t-1$ \\
	$C(q_t,q_{t-1})$	& Count of transitions in reverse. \\
	$C(q,w)$	& Count of words given a POS tag. \\
	$C(w,q)$	& Count of POS tags given a word.\\
	\hline
\end{tabular}
\end{center}
With this data, we can now perform different smoothing methods. The smoothing 
method is modular in our implementation, so as long as the counts are present, 
we can calculate the smoothed probabilities.

\section{Evaluation}
We performed evaluation for the POS tagger using two types of smoothing methods, 
Laplace smoothing using $B=1$ (or add-one smoothing) and Witten-Bell smoothing.  
The tagger performs smoothing for both conditional probabilities, $\P{w}{q}$ and 
$\P{q_t}{q_{t-1}}$.

\subsection{Different smoothing methods}
Using the \textbf{add-one smoothing}, we run 10-fold cross validation using
\texttt{sents.out}. The results are shown in Table \ref{addoneresult}.

\begin{table}
	\begin{center}
	\begin{tabular}{r | c c c}
	\hline
	Fold	&	Recall	&	Precision& $F1$ \\
	\hline
	1	 &	0.8645	&	0.7569	&	0.8618\\
	2	 &	0.8722	&	0.7702	&	0.8775\\
	3	 &	0.8726	&	0.7776	&	0.8656\\
	4	 &	0.8794	&	0.7599	&	0.8322\\
	5	 &	0.8924	&	0.7920	&	0.8626\\
	6	 &	0.8994	&	0.7603	&	0.8477\\
	7	 &	0.9110	&	0.7822	&	0.8523\\
	8	 &	0.8408	&	0.7705	&	0.8717\\
	9	 &	0.8689	&	0.7698	&	0.8462\\
	10	 &	0.8608	&	0.7811	&	0.8696\\
	\hline
Average	 &	0.8762	&	0.7721	&	0.85872\\
	\hline
	\end{tabular}
	\end{center}
	\caption{10-fold validation using add-one smoothing}
	\label{addoneresult}
\end{table}
The results show that add-one smoothing causes a bad recall rate. This means 
that some of the words are tagged erroneously, but generally distributed 
throughout the other tags, such that they do not affect the other recall 
measures much. This suggests that the probability distributions given after 
smoothing spread the density out over the words and part of speech tags too 
much.

Using \textbf{Witten-Bell smoothing}, there are fewer occurrences of such 
problems. This means, compared to add-one smoothing, Witten-Bell smoothing gives 
a better estimate of unseen instances of words given tags and tags given tags, 
resulting in a better overall prediction. The results are shown in Table 
\ref{wbresult}.
\begin{table}
	\begin{center}
	\begin{tabular}{r | c c c}
	\hline
	Fold	&	Recall	&	Precision	&	$F1$ \\
	\hline
	1		&	0.8541	&	0.8414		&	0.8719\\
	2		&	0.8715	&	0.8978		&	0.8697\\
	3		&	0.8595	&	0.9022		&	0.8723\\
	4		&	0.8645	&	0.8719		&	0.8837\\
	5		&	0.8890	&	0.8899		&	0.9024\\
	6		&	0.8890	&	0.8766		&	0.8802\\
	7		&	0.8924	&	0.8944		&	0.8877\\
	8		&	0.8479	&	0.8773		&	0.8520\\
	9		&	0.8468	&	0.8780		&	0.8736\\
	10		&	0.8677	&	0.8929		&	0.8769\\
	\hline
	Average	&	0.86824	&	0.8822		&	0.8770	\\
	\hline
	\end{tabular}

	\end{center}
	\caption{10-fold validation using Witten-Bell smoothing}
	\label{wbresult}
\end{table}
\subsection{Evaluating using \texttt{sents.devt}}

With these results, we selected the Witten-Bell smoothing technique, and trained 
the tagger using all instances in \texttt{sents.train}, and tested it using the 
\texttt{sents.devt} file. This gave us a recall, precision and $F1$ measure of 
0.8878, 0.9333 and 	0.9015 respectively.

So, how can we improve upon this? Looking at the breakdown given the parts of 
speech and the confusion matrix, we can know where our tagger breaks down, and 
suggest improvements to our preprocessing and smoothing steps. The following are 
some of the parts of speech tags that perform poorly:
\begin{description}
	\item[NNPS (Proper Noun, plural)] A considerable amount of the mistakes made 
		for NNPS were due to mis-classifications of these tokens as NNS and NNP.  
		One possible remedy for this is to be able to give a word that starts 
		with a capital letter and ending with an `s' a higher probability of 
		being an NNPS. This could be done as a special case within the smoothing 
		step given an unseen word. Another option would be to use finite state 
		transducers to break-down the morphology of the word.
	\item[RBR (Adverb, comparative)] Some words which should be tagged as RBR 
		are tagged as JJR. This makes sense due to the large number of overlap 
		in the type of words used. Looking at the tags that appear before and 
		after the occurences of RBR, they also seem to be rather similar. One 
		possibility to fix this would be to have higher order Markov chains for 
		the HMM.
	\item[LS (List item marker)] These are hard to distinguish from usual 
		mentions of numbers.  This is especially since one of our preprocessing 
		steps was to make all instances of digits the same. One possible way of 
		dealing with this would be not to modify the first tokens of a sentence, 
		since the LS tokens appear at the start pretty often.
\end{description}

\section{Conclusion}
Despite the use of smoothing, some other optimisations could still be made to 
further improve the performance of our tagger. These include using a Markov 
Chain with order greater than 1, accounting for plurality using transducers, and 
taking into account the position of the word in the sentence. These, however, 
fall outside the scope of the assignment.

We have, however, ascertained that using Witten-Bell smoothing for POS tagging 
works reasonably well compared to Add-one smoothing. We are also confident that 
our POS tagger works well, in most cases.
\end{document}
